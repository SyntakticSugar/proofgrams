\documentclass[11pt]{report}

% Document dimensions
\usepackage{geometry}
\geometry{top=1.5cm, bottom=1.5cm, textwidth=15cm}

% Math related packges.
\usepackage{amsmath}
\usepackage{cancel}

% Natural Deduction package
\usepackage{proof}
\usepackage{mdframed}

% Fix the header space: start at the top of the page.
\usepackage{hyperref}

% Import the necessary preamble for the document. 
\usepackage{../../../proofsPrograms}


\begin{document}
	
	
% Heading for the tutorial	
\begin{center}
	{\bf MATH230: $\lambda$-Calculus}
\end{center}
\begin{center}
	{\bf Church Encodings}
\end{center}


% Box with goals and relevent lecture notes.
\noindent\fbox{
	\parbox{\textwidth}{

		Key ideas
			\begin{itemize}
				\item Practice $\beta$-reduction, 
				\item Encode logic in $\lambda$-calculus, 
				\item Encode natural numbers in $\lambda$-calculus.
			\end{itemize}

		Relevant notes: Lambda Calculus Slides\\
		Relevant reading: Type Theory and Functional Programming, Simon Thompson
		
	\vspace{0.2cm}

	Hand in exercises: 1c, 2, 3a, 5\\ 
	{\bf Due following Friday @ 5pm.}
	}
}
% Discussion questions for tutor.
\newline
\vspace{0.5cm}

\noindent {\bf Discussion Questions}

\begin{itemize}


	\item Logical connectives are encoded as functions in the $\lambda$-calculus. They're intended to take in the Boolean $\lambda$-expressions TRUE and FALSE. However, without any extra structure to the $\lambda$-calculus, there is nothing stopping us writing:

	\vspace{0.2cm}
	\begin{itemize}
		\item[] NOT NOT NOT 
	\end{itemize}

	\vspace{0.2cm}

	Perform $\beta$-reduction on the above until the expression is in normal form. The strange answer you get suggests that we should not do this! To avoid this, extra \emph{type} structure is added to the $\lambda$-calculus; this amounts to saying only certain $\lambda$-expressions can be applied to others. If the language were typed sensibly, we would get a type error when trying to evaluate NOT NOT NOT.
	
\end{itemize}	

% New page for tutorial exercises.
\newpage
{\bf Tutorial Exercises}
\begin{enumerate}
	
	\item By substituting the explicit $\lambda$-expressions (as necessary) and performing $\beta$-reduction, show that the expressions below are $\beta$-equivalent to the Boolean values we should expect from the logical connectives involved.
	
		\begin{enumerate}
			\item NOT FALSE \vspace{0.5cm}
			\item OR TRUE FALSE \vspace{0.5cm}
			\item AND FALSE TRUE \vspace{0.5cm}
			\item IMPLIES FALSE TRUE 
		\end{enumerate} 

	{\bf Only expand those expressions necessary for each step.}
		
	\item Write down $\lambda$-expressions that represent the propositional binary connectives XOR, NAND, and NOR. Recall that these have the following truth tables.
			
	\vspace{0.5cm}

	\begin{center}
		$\begin{array}{c c c}

			\begin{array}{ c c | c }			
				P & Q & \text{XOR}(P,Q)\\
				\cline{1 - 3}
				T & T & F \\ 
				T & F & T \\ 
				F & T & T \\ 
				F & F & F
			\end{array} 
			& \hspace{0.5cm}
			\begin{array}{ c c | c }			
				P & Q & \text{NAND}(P,Q)\\
				\cline{1 - 3}
				T & T & F \\ 
				T & F & T \\ 
				F & T & T \\ 
				F & F & T
			\end{array} 
			& \hspace{0.5cm}
			\begin{array}{ c c | c }			
				P & Q & \text{NOR}(P,Q)\\
				\cline{1 - 3}
				T & T & F \\ 
				T & F & F \\ 
				F & T & F \\ 
				F & F & T
			\end{array} 
		\end{array}$
	\end{center}
	\vspace{0.2cm}

	\item By substituting the explicit $\lambda$-expressions (as necessary) and performing $\beta$-reduction, determine the normal forms of the following $\lambda$-expressions. 

	\begin{enumerate}
		\item SUCC ONE \vspace{0.2cm}
		\item SUM ONE ZERO \vspace{0.2cm}
		\item MULT TWO ZERO %\vspace{0.5cm}
	\end{enumerate}

	{\bf Only expand those expressions necessary for each step.}
	
	\newpage
	\begin{center}
		{\bf Compound Data with PAIR}
	\end{center}

	\item We have defined the following $\lambda$-expression to construct pairs of $\lambda$-expressions:
	
	$$\text{PAIR} = \lambda x. \ \lambda y. \ \lambda f. \ f \ x \ y$$

	The third input is a built-in place ready to take a selector:

	$$\text{FirST} = \lambda x. \ \lambda y. \ x \hspace{1cm} \text{SecoND} = \lambda x. \ \lambda y. \ y$$

	 Reduce these to normal form 

		\begin{enumerate}
			\item PAIR a b FST \vspace{0.2cm}
			\item PAIR a b SND \vspace{0.2cm}
			\item PAIR (PAIR a b) (PAIR c d) SND
		\end{enumerate}

	{\bf You may wish to make use of the app linked on learn to write your answers to the following problems.}

	\item Integers are solutions to equations $x + b = a$. We can represent natural number solutions using Church numerals. However, further abstractions are required to represent the negative solutions to such equations. 

	{\bf Example} The equation $x + 1 = 0$ has solution $x = -1$. One way to represent this in the $\lambda$-calculus is to use the pair $(0,1)$ which we interpret as $-1 = (0,1)$. 
	
	More generally $k = (a,b) = a - b$ represents the solution to $x + b = a$.

	\hspace{0.5cm} {\bf Such representations are not unique!} e.g. $0 = (0,0) = (1,1) = \dots$

	Write $\lambda$-expressions for arithmetic on integers as pairs of Church numerals.
		\begin{itemize}
			\item[] INT-SUM to calculate the sum of two integers.
			\item[] INT-MULT to calculate the product of two integers.
			\item[] INT-NEG(ative) to calculate the negative of an integer.
		\end{itemize}

	\item Rational numbers are solutions to equations of the form $bx = a$. Use PAIR to represent rational numbers in the $\lambda$-calculus and write $\lambda$-expressions to compute rational number arithmetic. 
		\begin{itemize}
			\item[] RAT-SUM to calculate the sum of two rational numbers.
			\item[] RAT-MULT to calculate the product of two rational numbers.
			\item[] RAT-REC(iprocal) to calculate the reciprocal of an integer.
		\end{itemize}

	\item Imaginary numbers can be represented as PAIRs of rational numbers.
	
	\item Real numbers are a more complicated issue.

\end{enumerate}
	
\end{document}