\documentclass[11pt]{report}

% Document dimensions
\usepackage{geometry}
\geometry{top=1.5cm, bottom=1.5cm, textwidth=15cm}

% Math related packges.
\usepackage{amsmath}
\usepackage{cancel}

% Natural Deduction package
\usepackage{proof}
\usepackage{mdframed}

% Fix the header space: start at the top of the page.
\usepackage{hyperref}

% Import the necessary preamble for the document. 
\usepackage{../../../proofsPrograms}


\begin{document}
	
	
% Heading for the tutorial	
\begin{center}
	{\bf MATH230: Tutorial XY}
\end{center}
\begin{center}
	{\bf Subtitle}
\end{center}

% Box with goals and relevent lecture notes.
\noindent\fbox{
	\parbox{\textwidth}{

		Key ideas and learning outcomes
			\begin{itemize}
				\item Write Turing machines for propositional logic, 
				\item Comment your code!
				\item Work on your assignment.
			\end{itemize}

		Relevant lectures: Turing machine slides\\
		Relevant reading: Linked in the Turing section on learn. 
		
	\vspace{0.2cm}

	Hand in exercises: 7 and 8
	}
}

% New page for tutorial exercises.
\noindent {\bf Tutorial Exercises}
\begin{enumerate}
	
	\item Negation. 
		
		\begin{itemize}
			\item[] Input: Single bit. 
			\item[] Output: Single bit, the negation of the input bit.
			\item[] Test: Input = 1 Output = 0
		\end{itemize}

	\item Disjunction of two bits. 

		\begin{itemize}
			\item[] Input: Two bits. 
			\item[] Output: Single bit, the disjunction of the input bits. 
			\item[] Test: Input = 10 Output = 1
		\end{itemize}		

	\item Conjunction of two bits. 

		\begin{itemize}
			\item[] Input: Two bits. 
			\item[] Output: Single bit, the conjunction of the input bit. 
			\item[] Test: Input = 10 Output = 0
		\end{itemize}		
		
	\item Implication. 

		\begin{itemize}
			\item[] Input: Two bits $(b_{1}b_{2})$. 
			\item[] Output: Single bit, representing $v(b_{1}\to b_{2})$. 
			\item[] Test: (i) Input = 10 Output = 0 (ii) Input = 00 Ouput = 1.
		\end{itemize}

		
	\item Bitwise-ORs of two binary strings. 

		\begin{itemize}
			\item[] Input: two space separated equal length binary strings. 
			\item[] Output: a binary string corresponding to the bitwise OR of input strings.
			\item[] Test: (i) Input = 10 10 Output = 10 (ii) Input = 10 01 Ouput = 11.
		\end{itemize}
	
	\newpage
	\item Bitwise-ANDs of two binary strings. 

		\begin{itemize}
			\item[] Input: Two space separated equal length binary strings. 
			\item[] Output: Binary string corresponding to the bitwise AND of the input strings.
			\item[] Test: (i) Input = 01 10 Output = 00 (ii) Input = 01 01 Ouput = 01.
		\end{itemize}	

	\item Disjunction of $n$-bits. 

		\begin{itemize}
			\item[] Input: One finite binary string.
			\item[] Output: Single bit corresponding to the disjunction of input bits.
			\item[] Test: Input = 1010101110 Output = 1.
		\end{itemize}
	
	\item Conjunction of $n$-bits. 

		\begin{itemize}
			\item[] Input: One finite binary string. 
			\item[] Output: Single bit corresponding to the conjunction of input bits.
			\item[] Test: Input = 1010101110 Output = 0.
		\end{itemize}		
	 
\end{enumerate}
	
\end{document}