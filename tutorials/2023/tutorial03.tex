\documentclass[11pt]{report}

% Document dimensions
\usepackage{geometry}
\geometry{top=1.5cm, bottom=1.5cm, textwidth=15cm}

% Math related packges.
\usepackage{amsmath}
\usepackage{cancel}

% Natural Deduction package
\usepackage{proof}

% Fix the header space: start at the top of the page.
\usepackage{hyperref}


\begin{document}
	
	
% Heading for the tutorial	
\begin{center}
	{\bf MATH230: Tutorial Three}
\end{center}
\begin{center}
	{\bf Natural Deductions: Classical Logic}
\end{center}


% Box with goals and relevent lecture notes.
\noindent\fbox{
	\parbox{\textwidth}{

		Key ideas
			\begin{itemize}
				\item Write some natural deductions using RAA,
				\item Prove LEM and DNE,
				\item Understand the induction step in the soundness theorem,
				%\item Appreciate the oddities of classical theorems. 
			\end{itemize}

		Relevant lectures: Lectures 6,7,8, and 9\\
		Relevant reading: \href{https://leanprover.github.io/logic_and_proof/index.html}{L$\exists\forall$N Chapters 3,4,5}  
		
	\vspace{0.2cm}

	Hand in exercises: 2a, 2b, 2e, 3a, 3b\\ 
	{\bf Due following Friday @ 5pm to the tutor, or lecturer.}
	}
}
% Discussion questions for tutor.
\newline
\vspace{0.5cm}

\noindent {\bf Discussion Questions}

\begin{enumerate}
	\item If $\Sigma_{1} \models \alpha \to \beta$ and $\Sigma_{2} \models \alpha$, then show $\Sigma_{1} \cup \Sigma_{2} \models \beta$.
	% This is one of the steps in the proof of the soundness theorem. Draw the rule of inference (MP) that this corresponds to. 
	
	\vspace{5cm} 
	
	\item $\lnot (A \land B) \vdash \lnot A \lor \lnot B$
	
\end{enumerate}

% New page for tutorial exercises.
\newpage
{\bf Tutorial Exercises}
\begin{enumerate}

	\item Make sure you have finished all of the minimal and intuitionistic natural deductions before doing this tutorial. It is more important that you understand those.
	
	\item \textbf{Classical derivations.} Provide natural deduction proofs of the following. All rules \emph{may} be required.
	
	\begin{enumerate}
		\item $\vdash   A \lor \neg A$ \hfill (law of excluded middle - RAA)
		\item $\neg\neg A \vdash   A$ \hfill (double negation elimination - RAA)
		\item $\neg( A \land  B) \vdash  \neg  A \lor \neg  B$ \hfill (De Morgan - RAA)
		\item $ A \to  B \vdash  \neg  A \lor  B$ \hfill (material implication - RAA)

		\item $\vdash  ( A \to  B) \lor ( B \to  C)$ \hfill (Challenge! - RAA)
		\item \(\vdash   A\land \neg  A \to  B\) \hfill (the \emph{paradox of entailment})
		\item \(\vdash   A\to B\lor\neg B\) \hfill (LEM)
		\item \(\vdash   A\to( B\to A)\) \hfill (weakening)
		\item \(\vdash  \neg A\to( A \to B)\) \hfill (a form of \emph{ex falso})
		\item $\vdash (\neg A\to A)\to A$ \hfill (RAA)
		\item $( A \to  B) \land ( C \to  D) \vdash ( A \rightarrow  D) \lor ( C \rightarrow  B)$ \hfill (Challenge! - RAA) 
		\item $\lnot ( A \to  B) \dashv\vdash  A \land \lnot B$ \hfill (Challenge! - RAA helps)
	\end{enumerate}

\item {\bf Soundness Proof}	Complete the induction step of the soundness theorem by answering the following. 

		\begin{enumerate}
			\item Implication introduction
			
			If $\Gamma \models \beta$, then show $\Gamma \backslash\{\alpha\} \models \alpha \rightarrow \beta$.
			
			\item Disjunction introduction
					
			If $\Gamma \models \alpha$, then show $\Gamma \models \alpha \lor \beta$.
			
			\item Disjunction elimination
			
			If $\Gamma_{1} \models \alpha \lor \beta$, $\Gamma_{2} \models \alpha \rightarrow \gamma$, and $\Gamma_{3} \models \beta \rightarrow \gamma$ then show the following is valid $\Gamma_{1}\cup\Gamma_{2}\cup\Gamma_{3} \models \gamma$.
			
			\item Conjunction introduction
			
			If $\Gamma_{1} \models \alpha$ and $\Gamma_{2} \models \beta$, then we need to show that $\Gamma_{1} \cup \Gamma_{2} \models \alpha \land \beta$. 
			
			\item Conjunction elimination
			
			If $\Gamma \models \alpha\land\beta$, then show $\Gamma \models \alpha$.

			\item Ex Falso Quodlibet
			
			If $\Gamma \models \bot$, then show $\Gamma \models \alpha$.
			
			\item Reductio Ad Absurdum
			
			If $\Gamma, \lnot \alpha \models \bot$, then show $\Gamma \models \alpha$.
	
		\end{enumerate}

	\item Ex Falso Quodlibet (The Law of Explosion) states that, for any propositions $P,Q$ we have the sequent $\{P \land \lnot P\} \ \vdash \ Q$. 
	
	Show that the rule of inference {\bf XF} can be \emph{derived} from {\bf minimal logic + RAA}. In this sense we might say classical logic is more powerful the intuitionistic logic.
	
	\newpage
	\item In class we discussed how classical logic can be obtained from intuitionistic logic by adding the following rule of inference \emph{reductio ad absurdum}: If $^{\Sigma}_{\bot}\mathcal{D}$ is a deduction of $\bot$ from $\Sigma$, then
	
	\begin{center}		
		$\begin{array}{c}		
		\infer[RAA]{\alpha}
		{\begin{array}{c} \hline \cancel{\lnot\alpha} \\ \Sigma \\ \mathcal{D} \\ \bot \end{array}}
		\end{array}$
	\end{center}
	
	is a derivation of $\alpha$ from the assumptions $\Sigma \backslash\{\lnot\alpha\}$.

	In this question we will explore this extension of logics in more detail. We will see that there are different methods for obtaining classical logic from intuitionistic logic.

		\begin{enumerate}
			\item Show that adding the rule of inference \emph{reductio ad absurdum} to intuitionistic propositional logic is equivalent to asserting that all the formulae $P \lor \lnot P$ for each proposition $P$ are theorems.
			
			% We have seen {\bf RAA} implies {\bf LEM} in class. Therefore, it suffices to show that we can derive the {\bf RAA} rule of inference from the assumption that $P \lor \lnot P$ is a theorem for every $P$.

			That is, if given a derivation $\mathcal{D}$ witnessing $\Sigma,\lnot P \vdash \bot$, then show that it can be extended (without using RAA) using the assumption of LEM i.e. $\vdash P \lor \lnot P$ to a derivation of $P$ with the assumption $\lnot P$ eliminated.

			\item Show that adding the rule of inference reductio ad absurdum to intuitionistic propositional logic is equivalent to adding the rule of inference of double negation elimination. 
			
			% We have seen {\bf RAA} implies {\bf DNE} in class. Therefore, it suffices to show that we can derive the {\bf RAA} rule of inference from the assumption that $\lnot \lnot P \to P$ is a theorem for every $P$. 

			That is, if given a derivation $\mathcal{D}$ witnessing $\Sigma,\lnot P \vdash \bot$, then show that it can be extended (without using RAA) using the assumption of DNE i.e. $\lnot \lnot P \ \vdash P$ to a derivation of $P$ with the assumption $\lnot P$ eliminated.
		\end{enumerate}

		Together these exercises show that the logics: 

			\begin{enumerate}
				\item Intuitionistic + RAA
				\item Intuitionistic + LEM 
				\item Intuitionistic + DNE 
			\end{enumerate}

		All have the same set of theorems. In this way, we see classical logic can be obtained from intuitionistic in a number of ways.

	\end{enumerate}	
\end{document}