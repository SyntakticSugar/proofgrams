\documentclass[11pt]{report}

% Document dimensions
\usepackage{geometry}
\geometry{top=1.5cm, bottom=1.5cm, textwidth=15cm}

% Math related packges.
\usepackage{amsmath}
\usepackage{cancel}

% Natural Deduction package
\usepackage{proof}
\usepackage{mdframed}

% Fix the header space: start at the top of the page.
\usepackage{hyperref}

% Import the necessary preamble for the document. 
\usepackage{../../../proofsPrograms}


\begin{document}
	
	
% Heading for the tutorial	
\begin{center}
	{\bf MATH230: Tutorial Eight}
\end{center}
\begin{center}
	{\bf Curry Howard Correspondence}
\end{center}


% Box with goals and relevent lecture notes.
\noindent\fbox{
	\parbox{\textwidth}{

		Key ideas
			\begin{itemize}
				\item Write well-typed programs simple type theory, 
				\item Interpret programs as proofs of propositions,
				\item Compare computation and proof-simplification.
			\end{itemize}

		Relevant lectures: Lectures x,y, and z\\
		Relevant reading: \href{https://leanprover.github.io/logic_and_proof/index.html}{L$\exists\forall$N Chapters 3,4} 
		
	\vspace{0.2cm}

	Hand in exercises: \\ 
	{\bf Due Friday @ 5pm to the tutor, or to lecturer.}
	}
}
% Discussion questions for tutor.
\newline
\vspace{0.5cm}

\noindent {\bf Discussion Questions}

\begin{enumerate}

	\item Show the following type is inhabited in the given local context:
	$$ f : P\rightarrow  Q \vdash  P\rightarrow ( P\times  Q)$$
	
	\vspace{3cm}

	\item Show the following type is inhabited in the given local context:
	$$ P\vdash \lnot \lnot  P$$
	
	\vspace{3cm}

	\item Show the following type is inhabited in the given local context:
	$$( P\times  Q) +  R \ \vdash \ ( P + R) \times (Q + R)$$

\end{enumerate}

% New page for tutorial exercises.
\newpage
{\bf Tutorial Exercises}

\begin{enumerate}
	
	\item This exercise breaks the proof that the following type is inhabited
	
	$$\vdash \ (P \to Q) \to (\lnot Q \to \lnot P)$$

	into steps to show what's happening when local variables are introduced. 

	\begin{enumerate}

		\item Using local variables (much like temporary hypotheses) introduce as many terms as possible into the local context (left of the turnstile $\vdash$) to get a new sequent. Proof that this type is inhabited in the local context will ultimately lead to a proof that the original type is inhabited. 
		
		(!) Remember $\lnot P \equiv P \to \bot$.

		\item Prove the following type is inhabited in the stated syntax 
		
		$$ f : P \to Q, \ g : \lnot Q, \ p : P \ \vdash \ \bot$$

		\item Extend the typing derivation above, through the use of $\lambda$ abstraction, to a proof that the original type is inhabited.

	\end{enumerate}
	
	\item Each sequent below defines a local context (terms to the left of the $\vdash$ turnstile) and a goal (the type on the right of the $\vdash$ turnstile) - Using the terms of the local context, show that the goal is inhabited.

	\begin{enumerate}
		\item $f : ( P\times Q) \rightarrow  R \vdash  P \rightarrow (Q \rightarrow  R)$
		\item $f : P \rightarrow (Q \rightarrow  R) \ \vdash \ (P\times Q) \rightarrow R$
		\item $t : \lnot P + \lnot Q \ \vdash \ \lnot(P\times  Q)$
		\item $f : \lnot(P + Q) \ \vdash \ \lnot P \times \lnot Q$
		\item $t : \lnot P \times \lnot Q \ \vdash \ \lnot(P + Q) $
		\item $f : P\rightarrow  Q, \  g : Q \rightarrow  R \ \vdash \ P \rightarrow R$
		\item $t : P + Q,\ \ f : P \to  R,\ \  g : Q \to S \vdash \ R + S$
		\item $f : P \to  R,\ \ g : Q \to  S,\ \ t : \neg R + \neg S \ \vdash \ \neg P + \neg Q$
		\item $p : P,\ \ f : \neg P \ \vdash \ \neg Q$
	   	\item $f : P \rightarrow Q, \  g : P\rightarrow \lnot Q \ \vdash \ \lnot P$
	\end{enumerate}		

	\item This exercise shows you an example of a general observation first made by William Tait, relating the simplifications of proofs and the process of computation in the $\lambda$-calculus. 
	
	Consider the following proof of the theorem $$\vdash \ A \land B \to B$$
	
	\begin{center}
		$\begin{array}{c}		
		  \infer[\to, 1]{A \land B \to B}
		  	{\infer[\land E_{L}]{B}
				{\infer[\land I]{B \land A}
					{\infer[\land E_{R}]{B}{\infer[1]{A \land B}{}} 
					\hspace{0.5cm}	&	\hspace{0.5cm}
					\infer[\land_{L}]{A}{\infer[1]{A \land B}{}}}}}
		\end{array}$
	  \end{center}

	  	\begin{enumerate}
			\item Determine the corresponding proof-object for this proof. 
			\item Why does the proof-object have a redex in it? 
			\item Perform the $\beta$-reduction on the proof object from (a).
			\item What proof does the reduced proof-object correspond to?
		\end{enumerate} 

\end{enumerate}	
\end{document}