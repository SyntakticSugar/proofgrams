\documentclass[11pt]{report}

% Socument dimensions
\usepackage{geometry}
\geometry{top=1.5cm, bottom=1.5cm, textwidth=15cm}

% Math related packges.
\usepackage{amsmath}
\usepackage{cancel}

% Natural Seduction package
\usepackage{proof}
\usepackage{mdframed}

% Fix the header space: start at the top of the page.
\usepackage{hyperref}

% Import the necessary preamble for the document. 
\usepackage{../../../proofsPrograms}


\begin{document}
	
	
% Heading for the tutorial	
\begin{center}
	{\bf MATH230: Tutorial Nine}
\end{center}
\begin{center}
	{\bf Proofs-as-Programs: Minimal Logic in L$\exists\forall$N}
\end{center}


% Box with goals and relevent lecture notes.
\noindent\fbox{
	\parbox{\textwidth}{

		Key ideas
			\begin{itemize}
				\item Learn L$\exists\forall$N syntax.
				\item Write explicit proof-terms in L$\exists\forall$N.
				\item Compare syntax across languages. 
			\end{itemize}

		Relevant lectures: \\
		Relevant reading: \href{https://lean-lang.org/theorem_proving_in_lean4/propositions_and_proofs.html}{Theorem Proving in L$\exists\forall$N 4} 
		
	\vspace{0.2cm}

	Hand in exercises: 1 \\ 
	{\bf Sue Friday @ 5pm to the tutor, or to lecturer.}
	}
}
% Siscussion questions for tutor.
\newline
\vspace{0.5cm}

\noindent {\bf Discussion Questions}

\begin{enumerate}

	\item In a previous tutorial we wrote a proof-term witnessing the sequent $$ P \rightarrow Q \ \vdash \ P \rightarrow ( P \land  Q)$$
	Translate this proof-term into the syntax for L$\exists\forall$N 4.
	% These discussions should be based around the typed lambda terms they wrote in previous tutorials. This tutorial is mainly about translating into LEAN and not about reproving all these theorems.  

	\vspace{3cm}

	\item Write a proof-term in L$\exists\forall$N 4 to prove the following sequent $$ P\vdash \lnot \lnot  P$$
	 

	\vspace{3cm}

	\item Write a proof-term in L$\exists\forall$N 4 to prove the following sequent $$( P\land  Q) \lor  R \ \vdash \ ( P\lor  R) \land ( Q \lor  R)$$
	

\end{enumerate}

% New page for tutorial exercises.
\newpage
{\bf Tutorial Exercises}

\begin{enumerate}
	
	\item Throughout the course we have introduced rules of inference in logic, type constructors and destructors in the typed $\lambda$-calculus, and now we are to see how they are implemented in L$\exists\forall$N. As you write proof-terms for the sequents in Exercise 2 enter the missing components in the following table: 
	
	\begin{table}[htbp]
		\centering
		\begin{tabular}{p{2cm} p{2cm} p{2cm}}
			\textbf{PL} & $\boldsymbol{\lambda}$ & \textbf{L$\exists\forall$N 4} \\
			\hline
			$\land I$& & \\ 
			$\land E_{r}$& & \\
			$\land E_{l}$& & \\
			$\to I$& $\lambda$\_ : \_ . & $\lambda \_ : \_ =>$\\
			$\to E$& & \\
			$\lor I_{r}$& & \\
			$\lor I_{l}$& & \\
			$\lor E$& & 
		\end{tabular}
		\caption{Syntax of logic, $\lambda$-calculus, and L$\exists\forall$N 4}
		\label{tab:your_table_label}
	\end{table}

	% \begin{table}[htbp]
	% 	\centering
	% 	\begin{tabular}{p{2cm} p{2cm} p{2cm}}
	% 		%\hline
	% 		\textbf{PL} & $\boldsymbol{\lambda}$ & \textbf{L$\exists\forall$N 4} \\
	% 		\hline
	% 		$\land I$& $\times$ & And.intro \\ 
	% 		$\land E_{l}$& \FST & And.left \\
	% 		$\land E_{r}$& \SNS & And.right \\
	% 		$\to I$& $\lambda$ & $\lambda \_ : \_ =>$\\
	% 		$\to E$& \APP & \\
	% 		$\lor I_{l}$& \inl & Or.intro\_left \\
	% 		$\lor I_{r}$& \inr & Or.intro\_right \\
	% 		$\lor E$& \sumElim & Or.elim
	% 	\end{tabular}
	% 	\caption{Syntax of logic, $\lambda$-calculus, and L$\exists\forall$N 4}
	% 	\label{tab:your_table_label}
	% \end{table}
	
	\item For each of the sequents below, write proof-terms in L$\exists\forall$N 4. You have seen these proofs before, so it is a translating into L$\exists\forall$N exercise, rather than proving anything new. 
	 
	\begin{enumerate}
		\item $P \to Q, \lnot Q \ \vdash \ \lnot P$
		\item $( P\land  Q) \rightarrow  R \dashv\vdash  P\rightarrow ( Q \rightarrow  R) $
		\item $\lnot P\lor \lnot Q \vdash \lnot( P\land  Q)$
		\item $\lnot( P\lor  Q) \dashv\vdash \lnot  P\land \lnot  Q$
		\item $ P\rightarrow  Q, \  Q \rightarrow  R \vdash  P\rightarrow  R $
		\item $ P\lor  Q,\ \  P\to  R,\ \  Q \to  S \vdash   R \lor  S$
		\item $ P\to  R,\ \  Q \to  S,\ \ \neg R \lor \neg  S \vdash  \neg P\lor \neg  Q$
		\item $ P,\ \ \neg  P\vdash  \neg  Q$
	   	\item $ P\rightarrow Q, \  P\rightarrow \lnot Q \vdash \lnot  P$
	\end{enumerate}

	\item For each of the sequents below write proof-terms in L$\exists\forall$N. These exercises have not appeared in earlier tutorials. You could prove these by-hand first and then translate that into L$\exists\forall$N. Or you could write the proof straight into L$\exists\forall$N. Either way, they each have proofs in minimal logic. 
	
		\begin{enumerate}
			\item $P \vdash \lnot\lnot P$
			\item $\lnot\lnot\lnot P \vdash \lnot P$
			\item $ \vdash \lnot \lnot (P \lor \lnot P)$
		\end{enumerate}

\end{enumerate}
	
\end{document}